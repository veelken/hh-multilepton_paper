\section{Introduction}
\label{sec:introduction}

During the eight years that have passed since the discovery~\cite{Higgs-Discovery_CMS,Higgs-Discovery_CMS_long,Higgs-Discovery_ATLAS} of the Higgs ($\PHiggs$) boson,
many of its properties have already been measured with remarkable precision~\cite{HIG-14-042,HIG-15-002,ATLAS_SpinCP,HIG-14-018,HIG-16-041}.
Among the most important of these properties is the $\PHiggs$ boson self-coupling, as the measurement of this coupling allows to reconstruct the Higgs potential,
which in turn will allow to verify that the mechanism that breaks the electroweak gauge symmetry is indeed the Higgs mechanism postulated by the Standard Model (SM).
The latter predicts the existence of a trilinear as well as of a quartic $\PHiggs$ boson self-coupling.
The quartic self-coupling is unfortunately not experimentally accessible at the LHC,
as its determination requires the measurement of the rate for triple $\PHiggs$ boson production, 
which is much too small to be measured at the LHC~\cite{de_Florian_2020}, 
even with the full luminosity of $3000\fbinv$ that is expected to be delivered by the upcoming high-luminosity LHC upgrade~\cite{HL-LHC-TDR}.
The trilinear self-coupling is experimentally accessible at the LHC and can be determined by measuring the rate for $\PHiggs$ boson pair ($\HH$) production.

In the SM, $\HH$ production proceeds via two different processes.
The leading order (LO) Feynman diagrams for the dominant ``gluon-fusion'' ($\ggHH$) process are shown in the upper half of Fig.~\ref{fig:Feynman_ggHH_and_qqHH_sm}.
The left diagram is referred to as ``triangle'' diagram and the one shown on the right as ``box'' diagram.
The amplitude of the triangle diagram varies proportional to the values of the $\PHiggs$ boson self-coupling, denoted by the symbol $\lambda$, 
and of the Yukawa coupling of the top quark, denoted by the symbol $\yt$,
while the amplitude of the box diagram does not depend on $\lambda$ and varies proportional to $\yt^{2}$.
The triangle and box diagrams interfere destructively with each other, 
causing the cross section for the $\ggHH$ process to exhibit a strong dependence on $\lambda$ and $\yt$.
We denote the values of $\lambda$ and $\yt$ with respect to the SM expectation for these couplings by the symbols $\kappal$ and $\kappat$, respectively.
The parameters $\kappal$ and $\kappat$ are referred to as ``coupling strength modifiers'' in the literature.
In the SM, $\kappal = 1$ and $\kappat = 1$, by definition.
The SM cross section for $\HH$ production via $\ggHH$ the process has been computed at next-to-next-to-leading (NNLO) accuracy in quantum chromodynamics (QCD)
and amounts to $31.05^{+1.40}_{-1.98}\fb$~\cite{Grazzini:2018hh}.
Effects due to the finite mass of the top quark are included up to next-to-leading (NLO) accuracy in the calculation.
The LO Feynman diagrams for the subdominant ``vector-boson-fusion'' ($\qqHH$) process are shown in the lower half of Fig.~\ref{fig:Feynman_ggHH_and_qqHH_sm}.
The  SM cross section for this process has been computed at next-to-next-to-next-to-leading order (N$^{3}$LO) in QCD
and amounts to $1.73 \pm 0.04\fb$~\cite{Dreyer:2018qbw}.
Similarly to the $\ggHH$ process, the interference between the different diagrams causes the cross section for the $\qqHH$ process to exhibit a strong dependence on the relevant couplings,
which in the case of the $\qqHH$ process are given by the five couplings $\lambda$, $\cW$, $\cWW$, $\cZ$, and $\cZZ$,
where the symbols $\cW$ ($\cZ$) and $\cWW$ ($\cZZ$) denote, respectively, the trilinear and quartic coupling of the $\PHiggs$ boson to the $\PW$ ($\PZ$) boson.
Assuming the couplings $\cW$ and $\cZ$ ($\cWW$ and $\cZZ$) to vary by the same amount with respect to their SM expectation 
and denoting the corresponding coupling strength modifier by the symbol $\kappaV$ ($\kappaVV$) leaves us with five free parameters
when analyzing $\HH$ production via the $\ggHH$ and $\qqHH$ process.
Due to the interference deviations of these five coupling strength modifiers from unity not only affect the rate of $\HH$ production,
but also the distribution of the $\HH$ signal in kinematic observables.
The distribution in $\mHH$, the mass of the $\PHiggs$ boson pair, is sensitive to variations in the $\PHiggs$ boson couplings $\kappal$ and $\kappat$ in particular,
as variations of these couplings affect the amplitudes of the triangle and box diagrams differently, thus altering their interference.
As the $\ggHH$ and $\qqHH$ production processes typically results in a broad distribution in $\mHH$,
they are referred to as non-resonant $\HH$ production.

The presence of as yet undiscovered particles, predicted by a variety of theories beyond the SM, may alter the $\HH$ production rate
as well as the distributions in kinematic observables.
If they exist, such particles are expected to give rise to loop diagrams similar to the ones shown in Fig.~\ref{fig:Feynman_ggHH_and_qqHH_sm}.
These loop contributions may enhance the $\HH$ production rate significantly,
as they occur at the same loop level as the triangle and box diagrams that govern $\HH$ production in the SM.
As none of these new particles have yet been discovered, however, one is lead to conclude that their mass is at the \TeV scale or higher,
in any case significantly above the scale of the electroweak symmetry breaking.
Their high mass allows to approximate the loop contributions of new particles by contact interactions of the $\PHiggs$ boson.
The approach is referred to as effective field theory (EFT) in the literature~\cite{Buchmuller:1985jz,Grzadkowski:2010es}.
We follow Ref.~\cite{Carvalho:2015ttv} and model the contact interactions relevant for $\HH$ production by the three couplings $\cg$, $\cgg$, and $\ctwo$,
referring to the contact interaction between two gluons and two $\PHiggs$ bosons, two gluons and one $\PHiggs$ boson, 
and between two top quarks and two $\PHiggs$ bosons, respectively.
The corresponding Feynman diagrams for $\HH$ production via the $\ggHH$ process are shown in Fig.~\ref{fig:Feynman_ggHH_eft}.
The couplings $\cg$, $\cgg$, and $\ctwo$ also alter the rate of $\HH$ production via the $\qqHH$ process.
We focus on the EFT interpretation of the $\ggHH$ process in this paper,
as we expect our analysis of $\HH$ production in final states with electrons, muons, and hadronically decaying $\PGt$ leptons
to have little sensitivity to the $\qqHH$ process, due to the small signal yield.

Besides the enhancement of the $\HH$ production rate from non-SM values of the $\PHiggs$ boson couplings or loop contributions of new particles,
an excess of $\HH$ signal over the rate of events expected in the SM may also results from decays of new heavy particles into pairs of SM $\PHiggs$ bosons.
Various theories beyond the SM postulate such decays, in particular:
two-Higgs doublet models~\cite{Craig:2013hca,Nhung:2013lpa},
composite Higgs models~\cite{Grober:2010yv,Contino:2010mh}, Higgs portal models~\cite{Englert:2011yb,No:2013wsa},
and models inspired by warped extra dimensions (WED)~\cite{Randall:1999ee}.
In models inspired by WED, the new heavy particles may either have spin $J=0$ or spin $J=2$,
and are referred to as ``radions'' and gravitons, respectively~\cite{Cheung:2000rw}.
In this paper, we focus on the case that the width of the heavy particles is negligible compared to the experimental resolution on $\mHH$.
The presence of heavy particles decaying to $\PHiggs$ boson pairs then manifests itself as a peak in the reconstructed $\mHH$ distribution.
We refer to this case as resonant $\HH$ production and we denote the heavy resonance by the symbol $\X$.

In this paper, we present the results of a search for non-resonant as well as resonant $\HH$ production
in final states with electrons, muons, and hadronically decaying $\PGt$ leptons. The latter are denoted by the symbol $\tauh$.
The search is based on proton-proton ($\Pp\Pp$) collision data recorded by the CMS experiment at $\sqrt{s} = 13\TeV$ center-of-mass energy
and corresponding to an integrated luminosity of $137.2\fbinv$.
Seven distinct final states, distinguished by the multiplicity of leptons and $\tauh$, are included in the analysis:
\zeroLeptonFourTau, \oneLeptonThreeTau, \twoLeptonssZeroTau, \twoLeptonTwoTau, \threeLeptonZeroTau, \threeLeptonOneTau, and \fourLeptonZeroTau,
where we have used the symbol $\ss$ to indicate that the two leptons are required to be of the same electric charge. 
These final states, to which we will also refer to as ``channels'', 
target $\HH$ signal events in which the $\PHiggs$ boson pair decays into either $\PW\PW\PW\PW$, $\PW\PW\PGt\PGt$, or $\PGt\PGt\PGt\PGt$.
The $\PW$ bosons and $\PGt$ leptons subsequently decay either leptonically or hadronically.
Dedicated multi-variate (MVA) methods are used to distinguish the $\PHiggs\PHiggs$ signal from backgrounds.

Phenomenological studies of the prospects for discovering the $\HH$ signal in the decay mode to $\PW\PW\PW\PW$ 
are documented in Refs.~\cite{Baur:2002rb,Baur:2002qd,Li:2015yia,Adhikary:2017jtu,Ren:2017jbg}.
The ATLAS collaboration has published the results of a search for non-resonant as well as resonant $\HH$ production in the same decay mode in Ref.~\cite{Aaboud:2018ksn},
based on $36.1\fbinv$ of $\Pp\Pp$ collision data recorded at $\sqrt{s} = 13\TeV$.
The phenomenological studies and also the ATLAS analysis focus on the final states $2\Plepton\ss$, $3\Plepton$, and $4\Plepton$.
The analysis published in this paper constitutes the first search for $\HH$ production in the decay modes $\PW\PW\PGt\PGt$ and $\PGt\PGt\PGt\PGt$.
Searches for $\HH$ production in $\Pp\Pp$ collisions at $\sqrt{s} = 7$, $8$, and $13\TeV$
have previously been performed by the CMS and ATLAS collaborations in the decay modes 
$\Pbottom\Pbottom\PGg\PGg$~\cite{Aad:2014yja,Khachatryan:2016sey,Sirunyan:2018iwt,Aaboud:2018ftw}, 
$\Pbottom\Pbottom\Pbottom\Pbottom$~\cite{Khachatryan:2015yea,Aad:2015uka,Aaboud:2018knk,Sirunyan:2018zkk,Sirunyan:2018tki}, 
$\Pbottom\Pbottom\PGt\PGt$~\cite{Aad:2015xja,Sirunyan:2017tqo,Sirunyan:2017djm,Aaboud:2018sfw}, 
$\Pbottom\Pbottom\PW\PW$~\cite{Sirunyan:2017guj}, 
and $\PW\PW\PGg\PGg$~\cite{Aaboud:2018ewm}.
Limits on $\HH$ production obtained from a combination of these analyses have been published by the CMS and ATLAS collaboration 
in Refs.~\cite{Aad:2015xja,Sirunyan:2017tqo,Sirunyan:2018ayu}.

The paper is structured as follows:
A a brief overview of the CMS detector is given in Section~\ref{sec:detector}.
In Section~\ref{sec:datasets}, we detail the datasets and simulation samples used.
The reconstruction of electrons, muons, $\tauh$, and jets,
as well as of different kinematic observables is detailed in Section~\ref{sec:eventReconstruction}.
This is followed by a description, in Section~\ref{sec:eventSelection}, of the event selection criteria applied in the seven channels that are included in the analysis.
The multi-variate methods that we use to distinguish the $\PHiggs\PHiggs$ signal from backgrounds are detailed in Section~\ref{sec:analysisStrategy}.
The estimation of backgrounds is described in Section~\ref{sec:backgroundEstimation},
followed by a description of the relevant systematic uncertainties in Section~\ref{sec:systematicUncertainties}.
The statistical procedure used to extract the $\HH$ production rate and the results obtained by our analysis are presented in Section~\ref{sec:results}.
The paper concludes with a summary in Section~\ref{sec:summary}.
