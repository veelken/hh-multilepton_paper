\section{Introduction}
\label{sec:introduction}

In the eight years that have passed since the discovery of the Higgs ($\PHiggs$) boson~\cite{Higgs-Discovery_CMS,Higgs-Discovery_CMS_long,Higgs-Discovery_ATLAS},
many of its properties have already been measured with remarkable precision~\cite{HIG-14-042,HIG-15-002,ATLAS_SpinCP,HIG-14-018,HIG-16-041}. %% TODO: UPDATE! - AWB 2021.01.15
One important property which remains mostly unknown is the $\PHiggs$ boson self-coupling.
A precise measurement of this coupling is important, as this measurement allows us to reconstruct the Higgs potential,
and thus verify that the mechanism which breaks electroweak gauge symmetry is indeed the Higgs mechanism postulated by the Standard Model (SM).
The latter predicts the existence of trilinear as well as of quartic $\PHiggs$ boson self-couplings.
The quartic self-coupling, measured via triple $\PHiggs$ boson production, is not experimentally accessible at the LHC~\cite{de_Florian_2020},
even with the full luminosity of 3000~\fbinv which should be delivered after the high-luminosity LHC upgrade~\cite{HL-LHC-TDR}.
The trilinear self-coupling however is experimentally accessible, and can be determined using measurements of $\PHiggs$ boson pair ($\HH$) production.

In the SM, $\HH$ production proceeds via two different processes.

The leading order (LO) Feynman diagrams for the dominant ``gluon fusion'' ($\ggHH$) process are shown in the upper half of Fig.~\ref{fig:Feynman_ggHH_and_qqHH_sm}.
The left ``triangle'' diagram amplitude varies proportionally to the values of the $\PHiggs$ boson self-coupling ($\lambda$)
and the Yukawa coupling of the top quark $\yt$,
while the right ``box'' diagram amplitude is insensitive to $\lambda$ and varies as $\yt^{2}$.
The triangle and box diagrams interfere destructively, 
so the $\ggHH$ cross section exhibits a strong dependence on $\lambda$ and $\yt$.
We denote the ratios of $\lambda$ and $\yt$ to their SM expectations as $\kappal$ and $\kappat$, respectively.
By definition, these ``coupling strength modifiers'' have values $\kappal = 1$ and $\kappat = 1$ in the SM.
The SM cross section for $\HH$ production via $\ggHH$ has been computed to be $31.05^{+1.40}_{-1.98}$~\fb
at next-to-next-to-leading (NNLO) accuracy in quantum chromodynamics (QCD)~\cite{Grazzini:2018hh}.
Effects due to the finite mass of the top quark are included up to next-to-leading (NLO) accuracy in the calculation.

The LO Feynman diagrams for the subdominant ``vector-boson-fusion'' ($\qqHH$) process are shown in the lower half of Fig.~\ref{fig:Feynman_ggHH_and_qqHH_sm},
where ``$\VH$'' refers to either a $\PW$ or $\PZ$ boson.
Its SM cross section has been computed at next-to-next-to-next-to-leading order (N$^{3}$LO) in QCD
and amounts to $1.73 \pm 0.04\fb$~\cite{Dreyer:2018qbw}.
The interference between different $\qqHH$ diagrams causes the cross section to exhibit a strong dependence on the five SM couplings
$\lambda$, $\cW$, $\cWW$, $\cZ$, and $\cZZ$,
where the $\cW$ ($\cZ$) and $\cWW$ ($\cZZ$) denote, respectively, the trilinear and quartic $\PHiggs$ boson couplings to the $\PW$ ($\PZ$) boson.

Assuming the two couplings $\cW$ and $\cZ$ ($\cWW$ and $\cZZ$) to vary by the same amount with respect to their SM expectation, 
and denoting the corresponding coupling strength modifier by the symbol $\kappaV$ ($\kappaVV$), we have in total four free parameters
when analyzing $\HH$ production via the $\ggHH$ and $\qqHH$ processes: $\kappal$, $\kappat$, $\kappaV$, and $\kappaVV$.
Due to interference, deviations of these five coupling strength modifiers from unity not only affect the rate of $\HH$ production,
but also the distribution of the $\HH$ signal in kinematic observables.
The $\PHiggs$ boson pair mass distribution ($\mHH$) is particularly sensitive to variations in $\kappal$ and $\kappat$,
as these couplings affect the triangle and box diagram amplitudes differently.
As $\ggHH$ and $\qqHH$ production typically result in a broad $\mHH$ distribution,
they are referred to as ``non-resonant''.

\begin{figure}[h!]
\setlength{\unitlength}{1mm}
\begin{center}
\begin{picture}(170,73)(0,0)
\put(  2.5, 43.0){\mbox{\includegraphics*[height=30mm]{figures/ggHH_triangle.pdf}}}
\put( 90.5, 43.0){\mbox{\includegraphics*[height=30mm]{figures/ggHH_box.pdf}}}
\put( -1.0,  2.0){\mbox{\includegraphics*[height=30mm]{figures/qqHH_c2v.pdf}}}
\put( 58.0,  2.0){\mbox{\includegraphics*[height=30mm]{figures/qqHH_cv.pdf}}}
\put(117.0,  2.0){\mbox{\includegraphics*[height=30mm]{figures/qqHH_lambda.pdf}}}
\put( 39.0, 40.0){\small (a)}
\put(120.0, 40.0){\small (b)}
\put( 19.0,  0.0){\small (c)}
\put( 78.0,  0.0){\small (d)}
\put(137.0,  0.0){\small (e)}
\end{picture}
\end{center}
\caption{
  LO Feynman diagrams for SM non-resonant $\HH$ production via gluon fusion (a,b)
  and via vector-boson-fusion (c,d,e).
}
\label{fig:Feynman_ggHH_and_qqHH_sm}
\end{figure}

The presence of undiscovered particles, predicted by a variety of theories beyond the SM, may alter the $\HH$ production rate
as well as distributions in kinematic observables.
If they exist, such particles are expected to give rise to loop diagrams similar to the ones shown in Fig.~\ref{fig:Feynman_ggHH_eft}.
These loop contributions may enhance the $\HH$ production rate significantly,
as they occur at the same loop level as the triangle and box diagrams that govern $\HH$ production in the SM.
As none of these new particles have been discovered, their mass is presumed to be at the \TeV scale or higher,
well above the scale of the electroweak symmetry breaking.
Their high mass allows loop contributions of new particles to be approximated by contact interactions with the $\PHiggs$ boson
using an effective field theory (EFT) approach~\cite{Buchmuller:1985jz,Grzadkowski:2010es}.
We follow Ref.~\cite{Carvalho:2015ttv} and parametrize the contact interactions relevant for $\HH$ production by the couplings $\cg$, $\cgg$, and $\ctwo$,
referring to the interactions between two gluons and two $\PHiggs$ bosons, two gluons and one $\PHiggs$ boson, 
and two top quarks and two $\PHiggs$ bosons, respectively.
The corresponding Feynman diagrams for $\ggHH$ production are shown in Fig.~\ref{fig:Feynman_ggHH_eft}.
Although $\cg$, $\cgg$, and $\ctwo$ also alter the $\qqHH$ production rate,
our analysis of $\HH$ decays to multiple leptons
has little sensitivity to the $\qqHH$ process, due to the small signal yield, 
and we thus concentrate our analysis of $\HH$ production in the EFT approach on the $\ggHH$ process in this paper.

\begin{figure}[h!]
\setlength{\unitlength}{1mm}
\begin{center}
\begin{picture}(170,32)(0,0)
\put(  0.5, 2.0){\mbox{\includegraphics*[height=30mm]{figures/ggHH_cg.pdf}}}
\put( 53.0, 2.0){\mbox{\includegraphics*[height=30mm]{figures/ggHH_cgg.pdf}}}
\put(111.0, 2.0){\mbox{\includegraphics*[height=30mm]{figures/ggHH_c2.pdf}}}
\put( 16.0, 0.0){\small (a)}
\put( 73.5, 0.0){\small (b)}
\put(138.0, 0.0){\small (c)}
\end{picture}
\end{center}
\caption{
  LO Feynman diagrams for non-resonant $\HH$ production via gluon fusion in an EFT approach
  where contact interactions between (a) two gluons and two $\PHiggs$ bosons, (b) two gluons and one $\PHiggs$ boson, 
  and (c) two top quarks and two $\PHiggs$ bosons are parametrized by the three couplings $\cg$, $\cgg$, and $\ctwo$, respectively.
}
\label{fig:Feynman_ggHH_eft}
\end{figure}

An excess of $\HH$ signal events may also result from decays of new heavy particles into pairs of SM $\PHiggs$ bosons.
Various theories beyond the SM postulate such decays, in particular
two-Higgs doublet models~\cite{Craig:2013hca,Nhung:2013lpa},
composite Higgs models~\cite{Grober:2010yv,Contino:2010mh}, Higgs portal models~\cite{Englert:2011yb,No:2013wsa},
and models inspired by warped extra dimensions (WED)~\cite{Randall:1999ee}.  %% TODO: do we want more up-to-date references? -AWB 2020.01.16
In models inspired by WED, the new heavy particles may have spin $0$ (``radions'') or spin $2$ (``gravitons'')~\cite{Cheung:2000rw}.
In this paper, the width of heavy particle $\X$ is assumed to be negligible compared to the experimental resolution on $\mHH$, 
so that ``resonant'' $\X \to \HH$ production would create a peak in the reconstructed $\mHH$ distribution at the mass $m_{\X}$ of the resonance.
The Feynman diagram for this process is shown in Fig.~\ref{fig:Feynman_ggHH_resonant}.

\begin{figure}[h!]
\setlength{\unitlength}{1mm}
\begin{center}
\includegraphics*[height=36mm]{figures/ggHH_resonant.pdf}
\end{center}
\caption{
  LO Feynman diagrams for resonant $\HH$ production.
}
\label{fig:Feynman_ggHH_resonant}
\end{figure}

In this paper, we present the results of a search for non-resonant as well as resonant $\HH$ production
in final states with multiple electrons or muons ($\Plepton$) or hadronically decaying $\PGt$ leptons ($\tauh$).
The search is based on proton-proton ($\Pp\Pp$) collision data recorded by the CMS experiment at $\sqrt{s} = 13\TeV$ center-of-mass energy
and corresponding to an integrated luminosity of $137.2$~\fbinv.
Seven distinct final states or ``categories'', distinguished by the multiplicity of leptons and $\tauh$, are included in the analysis:
\twoLeptonssZeroTau, \threeLeptonZeroTau, \fourLeptonZeroTau, \threeLeptonOneTau, \twoLeptonTwoTau, \oneLeptonThreeTau, and \zeroLeptonFourTau,
where the symbol $\ss$ indicates that the two leptons have the same electric charge. 
These final states target $\HH$ signal events in which the $\PHiggs$ boson pair decays into either \WWWW, \WWtt, or \tttt,
and the $\PW$ bosons and $\PGt$ leptons subsequently decay either leptonically or hadronically.
Multi-variate (MVA) methods are used to distinguish the $\HH$ signal from backgrounds,
with dedicated MVAs for SM, EFT, and resonant hypotheses in each category.

Phenomenological studies of the prospects for discovering the $\HH$ signal in the \WWWW decay mode
are documented in Refs.~\cite{Baur:2002rb,Baur:2002qd,Li:2015yia,Adhikary:2017jtu,Ren:2017jbg}.
The ATLAS collaboration has published the results of a search for non-resonant as well as resonant $\HH$ production in the same decay mode in Ref.~\cite{Aaboud:2018ksn},
based on $36.1$~\fbinv of $\Pp\Pp$ collision data recorded at $\sqrt{s} = 13\TeV$.
These analyses focus on the $\twoLeptonssZeroTau$, $\threeLeptonZeroTau$, and $\fourLeptonZeroTau$ final states.
This paper presents the first search for $\HH$ pairs decaying to \WWtt and \tttt.
Searches for $\HH$ production in $\Pp\Pp$ collisions at $\sqrt{s} = 7$, $8$, and $13\TeV$
have previously been performed by the CMS and ATLAS collaborations in the decay modes 
$\Pbottom\Pbottom\PGg\PGg$~\cite{Aad:2014yja,Khachatryan:2016sey,Sirunyan:2018iwt,Aaboud:2018ftw}, 
$\Pbottom\Pbottom\Pbottom\Pbottom$~\cite{Khachatryan:2015yea,Aad:2015uka,Aaboud:2018knk,Sirunyan:2018zkk,Sirunyan:2018tki}, 
$\Pbottom\Pbottom\PGt\PGt$~\cite{Aad:2015xja,Sirunyan:2017tqo,Sirunyan:2017djm,Aaboud:2018sfw}, 
$\Pbottom\Pbottom\PW\PW$~\cite{Sirunyan:2017guj}, 
and $\PW\PW\PGg\PGg$~\cite{Aaboud:2018ewm}.
Limits on $\HH$ production obtained from a combination of these analyses have been published by the CMS and ATLAS collaboration 
in Refs.~\cite{Aad:2015xja,Sirunyan:2017tqo,Sirunyan:2018ayu}.

The paper is structured as follows.
A brief overview of the CMS detector is given in Section~\ref{sec:detector}.
In Section~\ref{sec:datasets}, we detail the datasets and simulation samples used.
The reconstruction of electrons, muons, $\tauh$, and jets,
along with various kinematic observables is detailed in Section~\ref{sec:eventReconstruction}.
This is followed by a description, in Section~\ref{sec:eventSelection}, of the event selection criteria applied in the seven search categories.
The multi-variate methods used to distinguish the $\HH$ signal from backgrounds are detailed in Section~\ref{sec:analysisStrategy}.
The estimation of these backgrounds is described in Section~\ref{sec:backgroundEstimation},
followed by a description of the relevant systematic uncertainties in Section~\ref{sec:systematicUncertainties}.
The statistical procedure used to extract the $\HH$ production rate and the results obtained by our analysis are presented in Section~\ref{sec:results}.
The paper concludes with a summary in Section~\ref{sec:summary}.
