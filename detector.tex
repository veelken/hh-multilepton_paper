\section{The CMS detector}
\label{sec:detector}

The central feature of the CMS apparatus is a superconducting solenoid of $6$\unit{m} internal diameter, 
providing a magnetic field of $3.8$\unit{T}. 
Within the solenoid volume are a silicon pixel and strip tracker, 
a lead tungstate crystal electromagnetic calorimeter (ECAL), 
and a brass and scintillator hadron calorimeter (HCAL), 
each composed of a barrel and two endcap sections.
The silicon tracker measures charged particles within the pseudorapidity range $\abs{\eta} < 2.5$. 
The ECAL is a fine-grained hermetic calorimeter with quasi-projective geometry,
and is segmented into the barrel region of $\abs{\eta} < 1.48$ and in two endcaps that extend up to $\abs{\eta} < 3.0$.
The HCAL barrel and endcaps similarly cover the region $\abs{\eta} < 3.0$.
Forward calorimeters extend the coverage up to $\abs{\eta} < 5.0$.
Muons are detected within the range $\abs{\eta} < 2.4$ 
by gas-ionization detectors embedded in the steel flux-return yoke outside the solenoid.
Events of interest are selected using a two-tiered trigger system. 
The first level, composed of custom hardware processors, uses information from the calorimeters and muon detectors 
to select events at a rate of around $100$\unit{kHz} within a fixed latency of about $4$\mus~\cite{Sirunyan:2020zal}. 
The second level, known as the high-level trigger, 
consists of a farm of processors running a version of the full event reconstruction software optimized for fast processing, 
and reduces the event rate to around $1$\unit{kHz} before data storage~\cite{Khachatryan:2016bia}. 
A more detailed description of the CMS detector, 
together with a definition of the coordinate system used and the relevant kinematic variables, can be found in Ref.~\cite{Chatrchyan:2008zzk}. 
