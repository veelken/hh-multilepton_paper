\section{Results}
\label{sec:results}

The full set of data from 2016, 2017, and 2018 in all selection categories
is tested against $\HH$ production hypotheses from multiple models, described
in Section~\ref{sec:introduction}: the Standard Model prediction;
variations of the SM coupling constant modifiers
$\kappal$, $\kappat$, $\kappaV$, $\kappaVV$, and [TODO 5th?];
BSM effective couplings $\cg$, $\cgg$, and $\ctwo$;
and resonant production from the decay of spin-0 or
spin-2 particles with masses between 250 and $1000\GeV/c^{2}$.
In each case, the entire dataset is fit simultaneously to a model
composed of the background prediction and the $\HH$ signal hypothesis
under consideration.  The SM ``signal strength'' parameter is defined as
$\mu = \sigma(\HH)_{\textrm{best fit}} / \sigma(\HH)_{\textrm{SM}}$,
and modifies the expected signal yield by the same proportion in each category.
Variations in the $\kappa$ modifiers affect the signal yields in each cateogry
separately, and change the BDT output discriminant shape for $\HH$ events.
The twelve points spanning combinations of $\cg$, $\cgg$, and $\ctwo$ values in
the EFT phase space each carry different signal kinematics, so the $\HH$ cross
section $\sigma$ for each point is measured separately.  Similarly, signal efficiency
and BDT output vary dramatically for different resonant masses, so each mass and
spin value has its production cross section measured separately.  The SM signal
strength and $\kappa$ measurements are performed with $\ggHH$ and $\qqHH$ signal
MC simulation generated at NLO, using the SM ``node'' of the BDT training in
each event category to produce the final disciminator output distribution.
Each EFT point and resonant hypothesis has its own dedicated BDT node, and
is measured using a unique BDT output distribution in LO $\ggHH$ MC simulation.
[TODO: describe choice of BDT binning.]

The SM signal strength is measured using a profile likelihood test
statistic%%~\cite{Cowan:2010js},
with systematic uncertainties treated as
nuisance parameters $\theta$ in a modified frequentist
approach.%%~\cite{CMS-NOTE-2011-005}
The likelihood ratio for a fixed ``test'' signal strength value $\mu$ is:

\begin{linenomath}
\begin{equation*}
  \begin{aligned}
    q_{\mu}  &  = -2 \Delta \ln \mathcal{L} = \ln \frac{\mathcal{L}(\mathrm{data}|\mu,\hat{\theta}_{\mu})}{\mathcal{L}(\mathrm{data}|\hat{\mu},\hat{\theta})},
  \end{aligned}
\end{equation*}
\end{linenomath}

where $\hat{\mu}$ and $\hat{\theta}$ are the signal strength and nuisance
paramter values which give the maximum value of likelihood $\mathcal{L}$
for the given set of data, and $\hat{\theta}_{\mu}$ are the nuisance
parameter values which maximize $\mathcal{L}$ for the fixed $\mu$ value.
The 95\% confidence interval is bounded by values of $\mu$ for which
$q_{\mu} = 1.96$, close to 2 standard deviations from the global best-fit value.
The SM coupling strength modifiers and and BSM scenario cross sections
are measured by profiling values of $\kappa$ and $\sigma$, respectively,
relative to $\hat{\kappa}$ and $\hat{\sigma}$.
Theoretical and experimental uncertainties affecting the signal and
background yields or discriminator output distributions are fully
correlated across all years, event categories, and discriminant bins,
except as noted in Section~\ref{sec:systematicUncertainties}.


