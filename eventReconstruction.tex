\section{Event reconstruction}
\label{sec:eventReconstruction}

The CMS particle-flow (PF) algorithm~\cite{Sirunyan:2017ulk} reconstructs each individual particle in an event,
using an optimized combination of information from the various elements of the CMS detector.
The particles are subsequently classified into five mutually exclusive categories: 
electrons, muons, photons, and charged and neutral hadrons.
These particles are then combined to reconstruct hadronic $\PGt$ decays, jets, and the missing transverse momentum in the event.

Electrons are reconstructed within the geometric acceptance of the tracking detectors ($\abs{\eta} < 2.5$)
by combining information from the tracker and the ECAL~\cite{Khachatryan:2015hwa}.
They are initially identified using an MVA which distinguishes real electrons from hadrons, %% ~\cite{TODO: CMS EGamma}
along with loose requirements on track association with the collision vertex, and isolation from hadronic activity.
Electrons passing this intial selection are referred to as ``loose''.
In this analysis, events with leptons originating from hadron decay (``non-prompt''), or with
hadrons mis-identified as electrons (``fake''), consititute the largest source of background.
This motivates the use of an additional MVA, trained to select ``prompt'' electrons
from $\PW$, $\PZ$, and $\PGt$ decay, and reject fake or non-prompt electrons.
This MVA was previously used for measurements of $\ttH$ production in events with multiple leptons~\cite{Sirunyan:2020icl}.
It combines observables comparing measurements of the electron energy and direction in the tracker and the ECAL,
the compactness of the electron cluster, the bremsstrahlung emitted along the electron trajectory,
and the isolation of the electron with respect to other particles.
Electrons passing loose and tight thresholds of the MVA output are referred to as ``fakeable'' and ``tight'' electrons, respectively.
Only ``tight'' electrons are used to reconstruct signal candidate events, while data events with ``fakeable'' electrons which fail
the ``tight'' selection are used to estimate the contribution of fake and non-prompt electron backgrounds to the signal region.
The MVA selection thresholds in this analysis are lower than those in Ref.~\cite{Sirunyan:2020icl}, in order to increase
the efficiency for low-$\pt$ electrons which frequently appear in $\HH$ decays.
Electrons from photon conversions are suppressed by requiring that the track is missing no hits
in the innermost layers of the silicon tracker, and is not matched to a reconstructed conversion vertex.
In the \twoLeptonssZeroTau category, further electron selection criteria are applied,
requiring consistency among three independent measurements of the electron charge~\cite{Khachatryan:2015hwa}.

Muons are reconstructed by extrapolating tracks in the silicon tracker to hits in the gas-ionization
detectors embedded in the steel flux-return yoke outside the solenoid~\cite{Sirunyan:2018}.
To pass the initial ``loose'' identification requirement, muons must satisfy criteria related to
isolation and track proximity to the primary vertex, as well as track quality observables and
matching between the tracker and muon chambers. %% ~\cite{TODO: CMS Muon POG}
Additional requirements on the prompt vs.\ non-prompt muon identification MVA from Ref.~\cite{Sirunyan:2020icl} allow
us to select ``tight'' muons for signal candidate events, and ``fakeable'' muons for non-prompt background estimation.
Inputs to this MVA include energy deposits close to the muon in the ECAL and HCAL,
the hits and track segments reconstructed in the gas-ionization detectors located outside the solenoid,
the quality of the spatial matching between the track segments reconstructed in the silicon tracker and in the gas-ionization detectors,
and the isolation of the muon with respect to other particles.
Again, lower MVA selection thresholds compared to Ref.~\cite{Sirunyan:2020icl} bring higher efficiency for the $\HH$ signal.
In the \twoLeptonssZeroTau channel, the uncertainty in the curvature of the muon track is required to be less than $20\%$
to ensure a high-quality charge measurement.

Hadronic decays of $\PGt$ leptons are identified using the ``hadrons-plus-strips'' (HPS) algorithm~\cite{Sirunyan:2018pgf}.
This algorithm classifies individual hadronic decay modes of the $\PGt$ lepton
by combining charged hadrons from the PF reconstruction with neutral pions.
The latter are reconstructed by clustering electrons and photons into rectangular strips,
which are narrow in $\eta$ but wide in the $\phi$ direction.
The spread in $\phi$ accounts for photons originating from neutral pion decays
which convert into electron-positron pairs when traversing the silicon tracker.
These electrons and positrons are bent in opposite directions in $\phi$ by the magnetic field,
and may further emit bremsstrahlung photons while traversing the tracker.
The decay modes considered in this analysis are
$\PGt^{-} \to \Ph^{-}\Pnut$, $\PGt^{-} \to \Ph^{-}\Ppizero\Pnut$, $\PGt^{-} \to \Ph^{-}\Ppizero\Ppizero\Pnut$, 
$\PGt^{-} \to \Ph^{-}\Ph^{+}\Ph^{-}\Pnut$, and $\PGt^{-} \to \Ph^{-}\Ph^{+}\Ph^{-}\Ppizero\Pnut$
(plus the charge-conjugate decays),
where $\Ph^{-}$ and $\Ph^{+}$ denote charged pions or kaons.
The``DeepTau'' algorithm~\cite{CMS-DP-2019-033} identifies $\tauh$ objects
using a convolutional artificial neural network (ANN)~\cite{lecun1989}
with $42$ high-level observables as input, plus low-level information obtained from the silicon tracker, the electromagnetic and hadronic calorimeters, and the muon detectors.
The former comprise the $\pt$, $\eta$, $\phi$, and mass of the $\tauh$ candidate, the reconstructed $\tauh$ decay mode,
its isolation with respect to charged and neutral particles,
and the estimated distance that the $\Pgt$ lepton traverses between its production and decay.
The low-level information quantifies the particle activity within two $\eta \times \phi$ grids, centered on the direction of the $\tauh$ candidate:
an ``inner'' grid of size $0.2 \times 0.2$, filled with $0.02 \times 0.02$ cells,
and an ``outer'' grid of size $0.5 \times 0.5$ (partially overlapping with the inner grid), with $0.05 \times 0.05$ cells.
The $\tauh$ considered in this analysis are required to satisfy the conditions $\pt > 20\GeV$ and $\abs{\eta} < 2.3$,
and to pass a loose or tight selection on the output of the ANN, designating ``fakeable'' and ``tight'' $\tauh$, respectively.
Additional discriminants are used to separate $\tauh$ from electrons and muons.

Jets ($\jet$) are reconstructed with the infrared and collinear safe anti-\kt algorithm~\cite{Cacciari:2008gp, Cacciari:2011ma},
using PF reconstructed particles as input,
and serve to identify $\PHiggs \to \PW\PW \to \jet\jet\Plepton\Pnu$ decays in this analysis.
Jets reconstructed with distance parameters of $0.4$ (``AK4'') and $0.8$ (``AK8'') are both used:
two AK4 jets to reconstruct the two quarks from low-$\pt$ $\PW$ decays, or a single AK8 jet to reconstruct high-$\pt$
$\PW$ decays, where the quarks are collimated.
As the leptons produced in $\PHiggs \to \PW\PW$ decays with boosted $\PW$ bosons tend to be close to the jets,
leptons are removed from the AK8 jet reconstruction.
%% %% I don't think we need to describe the full jet calibration and validation procedure.
%% %% Jets are our least-important object, and we don't go into nearly this much detail for electrons or muons. - AWB 11.02.2021
%% The momentum of AK4 and AK8 jets is determined as the vectorial sum of all particle momenta in the jet,
%% and is found from simulation to be, on average, within $5$ to $10\%$ of the true momentum over the whole $\pt$ spectrum and detector acceptance.
%% Additional $\Pp\Pp$ interactions within the same or nearby bunch crossings (pileup) can contribute additional tracks and calorimetric energy depositions to the jet momentum.
%% To mitigate this effect, charged particles identified to be originating from pileup vertices are discarded and an offset correction is applied to correct for remaining contributions.
%% Jet energy corrections are derived from simulation to bring the measured response of jets to that of particle level jets on average.
%% In situ measurements of the momentum balance in dijet, $\text{photon} + \text{jet}$, $\PZ + \text{jet}$, and multijet events
%% are used to account for any residual differences in the jet energy scale between data and simulation~\cite{Khachatryan:2016kdb}.
After correcting for energy deposits from additional $\Pp\Pp$ interactions (pileup) and applying energy calibration,
the jet energy resolution amounts to $15$--$20\%$ at $30\GeV$, $10\%$ at $100\GeV$, and $5\%$ at $1\TeV$~\cite{Khachatryan:2016kdb}.
%% Additional selection criteria are applied to each jet to remove jets potentially dominated by anomalous contributions from various subdetector components or reconstruction failures.
This analysis considers jets reconstructed in the region $\abs{\eta} < 2.4$
with $\pt > 25\GeV$ (for AK4 jets) or $\pt > 170\GeV$ (for AK8 jets).
Additional criteria are applied to AK8 jets to specifically select those from boosted hadronic $\PW$ boson decays~\cite{Sirunyan:2019quj}.

Events containing AK4 jets identified with the hadronization of bottom quarks are vetoed in this analysis.
The ``DeepJet'' algorithm~\cite{CMS-DP-2017-013}
exploits observables related to the long lifetime of $\Pbottom$ hadrons
and the higher particle multiplicity and mass of $\Pbottom$ jets compared to light quark and gluon jets.
%% The properties of charged and neutral particle constituents of the jet as well as of secondary vertices reconstructed within the jet
%% are used as inputs to a convolutional ANN.
Both ``loose'' and ``medium'' b-tag selections on the ``DeepJet'' output are employed in this analysis,
corresponding to $\Pbottom$ jet selection efficiencies of $84\%$ and $70\%$, respectively,
while the mistag rates for light-quark and gluon jets are $11\%$ and $1.1\%$.

The missing transverse momentum vector $\ptvecmiss$ is computed as the negative vector sum of the transverse momenta of all the particles reconstructed by the PF algorithm in an event, 
and its magnitude is denoted as $\ptmiss$~\cite{Sirunyan:2019kia}. 
The $\ptvecmiss$ is modified to account for corrections to the energy scale of the reconstructed jets in the event. 
A linear discriminant, denoted by the symbol $\metLD$,
is employed to remove backgrounds in which the reconstructed $\ptmiss$ arises from resolution effects.
The discriminant is defined by the relation $\metLD = 0.6 \ptmiss + 0.4 \metHT$,
where $\metHT$ corresponds to the magnitude of the vector $\pt$ sum of electrons, muons, $\tauh$, and jets selected in this analysis~\cite{Sirunyan:2018shy}.
