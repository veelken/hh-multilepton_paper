\section{Data samples and Monte Carlo simulation}
\label{sec:datasets}

The data analyzed represent a total of $137.2$~\fbinv of proton-proton collisions
collected by the CMS detector in Run~2, which were certified to have all detector components
working properly: $35.9$~\fbinv in 2016, $41.5$~\fbinv in 2017, and $59.7$~\fbinv in 2018.
The two-tier CMS ``trigger'' system selects one event of interest out of every thirty
thousand collision events to save for closer examination, based on the presence of
specific combinations of reconstructed objects~\cite{Khachatryan:2016bia,Sirunyan:2020zal}.  This analysis
uses triggers requiring one or more electrons, muons, or $\tauh$
leptons to be associated with the primary collision vertex. The exact triggers and their
thresholds varied slightly from year to year, due to changing luminosity and detector
conditions, as well as improvements to the trigger algorithms.  The trigger $\pt$ and
$\eta$ requirements for each year are shown in Table~\ref{tab:triggers}.  All
triggers included identification and isolation requirements on the leptons and $\tauh$ as well.

\begin{table}[!h]
\begin{center}
\begin{tabular}{|l|l|}

\hline
Trigger            & Object selection (electron $\Pe$, muon $\Pgm$, and hadronic tau $\tauh$) \\
\hline
Single $\Pe$       & $\pt(\Pe) > 27 / 32 - 35 / 32\GeV$ (2016/17/18) \\
Single $\Pgm$      & $\pt(\Pgm) > 22 - 24 / 24 - 27 / 24\GeV$ (2016/17/18) \\ 
\hline
Double $\Pe$       & $\pt(\Pe) > 23, 12\GeV$ \\
$\Pe$ + $\Pgm$     & $\pt(\Pe) > 23\GeV$, $\pt(\Pgm) > 8\GeV$ \\
$\Pgm$ + $\Pe$     & $\pt(\Pgm) > 23\GeV$, $\pt(\Pe) > 8 / 12 / 12\GeV$ (2016/17/18) \\
Double $\Pgm$      & $\pt(\Pgm) > 17, 8\GeV$ \\
$\Pe$ + $\tauh$    & $\pt(\Pe) > 24\GeV$, $\pt(\tauh) > 20 - 30 / 30 / 30$~GeV, $\abs{\eta(\Pe, \tauh)} < 2.1$ (2016/17/18) \\
$\Pgm$ + $\tauh$   & $\pt(\Pgm) > 19 / 20 / 20\GeV$, $\pt(\tauh) > 20 / 27 / 27$~GeV, $\abs{\eta(\Pgm, \tauh)} < 2.1$ (2016/17/18) \\
Double $\tauh$     & $\pt(\tauh) > 35 - 40, 35 - 40\GeV$, $\abs{\eta(\tauh)} < 2.1$  \\
\hline
Triple $\Pe$       & $\pt(\Pe) > 16, 12, 8\GeV$ \\
Two $\Pe$ + $\Pgm$ & $\pt(\Pe) > 12, 12\GeV$, $\pt(\Pgm) > 8\GeV$ \\
Two $\Pgm$ + $\Pe$ & $\pt(\Pgm) > 9, 9\GeV$, $\pt(\Pe) > 9\GeV$ \\
Triple $\Pgm$      & $\pt(\Pgm) > 12, 10, 5\GeV$ \\
\hline

\end{tabular}
\end{center}
\caption{
  Selection requirements on \pt and $\eta$ for lepton-triggered events used in this analysis.  All
  leptons were additionally required to pass identification, isolation, and vertexing requirements.
}
\label{tab:triggers}
\end{table}

%% TODO: how were events "cross-cleaned" between datasets? (AWB 11.01.21)

Monte Carlo (MC) simulated samples were used to model $\HH$ signal events, and a wide range
of SM background processes which decay to leptons or $\tauh$.  Background MC samples
include those producing a single $\PW$ or $\PZ$ boson, two bosons ($\PW\PW$, $\PW\PZ$, $\PZ\PZ$, $\PW\Pgg$, and $\PZ\Pgg$),
three bosons ($\PW\PW\PW$, $\PW\PW\PZ$, $\PW\PZ\PZ$, $\PZ\PZ\PZ$, and $\PW\PZ\Pgg$), a single Higgs boson (via gluon fusion,
vector boson fusion, or associated production with a $\PW$ or $\PZ$ boson), a single top quark,
a top anti-top quark pair ($\ttbar$), and top quarks associated with one or more bosons ($\ttW$,
$\ttZ$, $\ttH$, $\tHq$, and $\tHW$).  Most samples, including the dominant irreducible $\PW\PZ$ and $\PZ\PZ$
backgrounds, were generated at next-to-leading-order (NLO) and scaled to cross sections
computed at next-to-next-to-leading-order (NNLO).

A variety of $\HH$ signal samples were generated, including $\ggHH$ and $\qqHH$
where each of the $\PHiggs$ bosons decays to either $\PW\PW$, $\PZ\PZ$, or $\PGt\PGt$.
Non-resonant $\HH$ MC samples include SM production, ``box-only'' $\HH$ events with
the trilinear Higgs self-coupling ($\kappa_{\lambda}$) set to zero, and $12$ benchmark scenarios
in the EFT approach.  These benchmarks represent different combinations of
$\kappal$, $\kappat$, $\ctwo$, $\cg$, and $\cgg$ values which correspond
to regions of similar LO kinematics, described in~\cite{Carvalho:2015ttv}.  The parameter
values are shown in Table~\ref{tab:HH_benchmarks}.  The EFT benchmark samples were
generated at LO but are reweighted to NLO accuracy using the results of Ref.~\cite{Buchalla:2018yce}.

\begin{table}[!h]
\begin{center}
\begin{tabular}{|l|c|c|c|c|c|}

\hline
Benchmark & $\kappal$  & $\kappat$ & $\ctwo$ & $\cg$ &  $\cgg$ \\
\hline
1         & 7.5        & 1.0       & -1.0    & 0.0   & 0.0  \\
2         & 1.0        & 1.0       &  0.5    & -0.8  & 0.6  \\
3         & 1.0        & 1.0       & -1.5    & 0.0   & -0.8 \\
4         & -3.5       & 1.5       & -3.0    & 0.0   & 0.0  \\
5         & 1.0        & 1.0       & 0.0     & 0.8   & -1.0 \\
6         & 2.4        & 1.0       & 0.0     & 0.2   & -0.2 \\
7         & 5.0        & 1.0       & 0.0     & 0.2   & -0.2 \\
8         & 15.0       & 1.0       & 0.0     & -1.0  & 1.0  \\
9         & 1.0        & 1.0       & 1.0     & -0.6  & 0.6  \\
10        & 10.0       & 1.5       & -1.0    & 0.0   & 0.0  \\
11        & 2.4        & 1.0       & 0.0     & 1.0   & -1.0 \\
12        & 15.0       & 1.0       & 1.0     & 0.0   & 0.0  \\
\hline
SM        & 1.0        & 1.0       & 0.0     & 0.0   & 0.0  \\
Box-only  & 0.0        & 1.0       & 0.0     & 0.0   & 0.0  \\
\hline

\end{tabular}
\end{center}
\caption{
  Parameter values for $\kappal$, $\kappat$, $\ctwo$, $\cg$, and
  $\cgg$ in MC samples modeling $12$ EFT benchmark scenarios, plus SM
  and ``box-only'' ($\lambda = 0$) $\HH$ production.
}
\label{tab:HH_benchmarks}
\end{table}

Resonant $\HH$ production was simulated at LO for both spin-0 (radion) and spin-2 (graviton)
scenarios with masses of $250$, $260$, $270$, $280$, $300$, $320$, $350$, $400$, $450$, $500$, $550$, $600$,
$650$, $700$, $750$, $800$, $850$, $900$, and $1000\GeV$.

All MC samples were generated using $\textsc{MadGraph5\_aMCatNLO}$~\cite{Alwall:2014hca} or
$\textsc{POWHEG}$ $v2$~\cite{Nason:2004rx,Frixione:2007vw,Alioli:2010xd}, with parton shower, hadronization
processes, and $\PGt$ lepton decays modeled by
$\textsc{PYTHIA}$~\cite{PYTHIA_CUETP8M1tune_CMS,PYTHIA_CUETP8M2tune_CMS,Sirunyan:2019dfx,PYTHIA_MonashTune}.
The CMS apparatus was modeled in detail using $\textsc{geant4}$~\cite{Agostinelli:2002hh}, and simulated events
were reconstructed using the same procedure as in data.  Additional collisions were generated with $\textsc{PYTHIA}$ and
overlaid on all MC events, with weights used to match the collision multiplicity distribution in data.
