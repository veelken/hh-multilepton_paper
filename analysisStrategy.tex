\section{Analysis strategy}
\label{sec:analysisStrategy}

The rate of the $\HH$ signal is extracted through a maximum-likelihood (ML) fit to
the output of boosted decision tree (BDT) classifiers~\cite{Breiman:1984jka},
which are trained to discriminate the $\HH$ signal from backgrounds.
Three such classifiers are trained for each of the seven search categories,
targeting non-resonant $\HH$ production and resonant $\HH$ production from
the decay of heavy particles of spin $0$ and of spin $2$, respectively.
The binning for each BDT output distribution is chosen such that the same
amount of background is contained in each bin, with the number of bins
determined by the condition that the statistical uncertainty in each bin does
not exceed $XX\%$ relative to the background contribution expected in that bin.

The observables used as inputs to the BDT classifiers are:
the $\pt$ and $\eta$ of reconstructed electrons, muons, and $\tauh$;
the angular separation between $\Pe$, $\PGm$, and $\tauh$,
quantified by the distance $\Delta R = \sqrt{(\eta_{1} - \eta_{2})^{2} + (\phi_{1} - \phi_{2})^{2}}$ between any pair of particles in $\eta$-$\phi$ space,
and the angular separation of the $\Pe$, $\PGm$, and $\tauh$ to the nearest jet;
the number of jets in the event;
the missing transverse momentum, quantified by the discriminant $\metLD$ described in Section~\ref{sec:eventReconstruction};
the scalar sum in $\pt$ of reconstructed $\Pe$, $\PGm$, $\tauh$, and jets;
the ``visible'' mass of the $\PHiggs$ boson pair, given by the mass of the system of reconstructed $\Pe$, $\PGm$, and $\tauh$,
and the ``full'' mass of the $\HH$ system, including neutrinos, reconstructed the algorithm described in Ref.~\cite{Ehataht:2018nql}.
The latter algorithm targets $\HH$ signal events decaying to \tttt and thus works best in the search categories \zeroLeptonFourTau, \oneLeptonThreeTau, and \twoLeptonTwoTau,
where the signal contribution from the decay modes \WWWW and \WWtt is small compared to the contribution from the \tttt decay mode.

These observables are complemented by further inputs, which allow the BDT to learn that the distributions in these observables for $\HH$ signal events
change as function of model parameters:
the $\PHiggs$ boson couplings $\lambda$, $\yt$, $\cW$, $\cZ$, $\cWW$, $\cZZ$, $\cg$, $\cgg$, and $\ctwo$ in case of non-resonant $\HH$ production
and the mass of the heavy particle $\X$ in case of resonant $\HH$ production.
As previously mentioned, this analysis has little sensitivity to the $\qqHH$ process and thereby to the couplings$\cW$, $\cZ$, $\cWW$, and $\cZZ$.
We thus limit the analysis of non-resonant $\HH$ production to the five couplings $\lambda$, $\yt$, $\cg$, $\cgg$, and $\ctwo$.
The variations in these couplings are analyzed in terms of the $12$ benchmark scenarios defined in Ref.~\cite{Carvalho:2015ttv}.
The BDT for non-resonant $\HH$ production is trained on samples of $\PHiggs$ boson pairs produced via SM $\HH$ production
and on the samples corresponding to the $12$ different benchmark scenarios in the EFT approach.
The type of sample is indicated using thirteen binary inputs to the BDT.
The BDT classifiers used for the analysis of resonant $\HH$ production are trained on the full set of masses $m_{\X}$ described in Section~\ref{sec:datasets}
and the mass $m_{\X}$ of the resonance is used as a real-valued input to the BDT.

The training is performed using the $\textsc{XGBoost}$ algorithm~\cite{Chen:2016btl},
which is interfaced to the $\textsc{Scikit-learn}$ machine learning library~\cite{scikit-learn}.
The parameters of the BDT training (so-called ``hyperparameters'') are optimized using the procedure described in Ref.~\cite{Tani:2020dyi}.

Various imprecisely known or measured effects affect the event yields of the $\HH$ signal and of the background processes that are modeled using the MC simulation,
as well as the shape of the distribution in the discriminating observable.
These effects are collectively referred to as systematic uncertainties and are included as nuisance parameters $\theta$ into the ML fit.

Statistical uncertainties on the distributions in the discriminating observable for the $\HH$ signal and for background processes
are accounted for in the ML fit by the approach detailed in Ref.~\cite{Barlow:1993dm}.



