\section{Event selection}
\label{sec:eventSelection}

Events are selected with the aim of maximizing the acceptance for HH decays to $\WWWW$, $\WWtt$, and
$\tttt$, while simultaneously rejecting the overwhelming backgrounds, which arise from multijet, single $\PW$
boson, $\PZ$ boson, and $\ttbar$ production.  To achieve this, we require multiple
reconstructed leptons $\lep$ (electrons or muons) or hadronically decaying taus
\tauh to be associated with the collision vertex in each event.  The leptons and $\tauh$ may originate
from the decay of a $\PW$ boson or from a $\Pgt$ decay.  Signal candidate events
are sub-divided into seven mutually exclusive categories based on lepton and $\tauh$
multiplicity: two same-sign leptons with less than two hadronic taus (\llss),
three leptons with no hadronic taus (\lllnot), at least four leptons (\llll),
three leptons with at least one additional hadronic tau (\lllt), two
leptons with at least two hadronic taus (\lltt), one lepton with at least
three hadronic taus (\lttt), or four or more hadronic taus with no leptons (\noltttt).
The event category is determined using lepton and $\tauh$ multiplicity based on the
``fakeable'' identification criteria described in Section~\ref{sec:eventReconstruction}.

To reduce background contamination from processes with top quarks, events with
at least one AK4 jet passing the ``medium'' b-tag identification, or at least two
jets passing the ``loose'' b-tag criteria, are discarded. 
Leptons originating from low-mass Drell-Yan production, from decays of $\PJpsi$ and $\PUpsilon$ meson,
from cascade decays of bottom quarks, and from photon conversions
are removed by vetoing events containing a pair of ``loose'' leptons of mass $m(\leplep) < 12\GeV$.
To eliminate overlap with events selected in the $\HH \to \Pbottom\APbottom\PZ\PZ$ decay channel~\cite{HIG-20-004}, no event in the
non-\tauh categories (\llss, \lllnot, and \llll) may contain two same-flavor,
opposite-charge ``loose'' lepton pairs with a combined mass less than $140\GeV$.
In addition, to reduce the \Zll background, these categories along with
\lltt exclude events where any same-flavor opposite-charge pair of ``loose''
leptons has a mass between $81$ and $101\GeV$.

The leptons and \tauh selected in any of the seven event categories must pass the ``tight'' selection 
described in Section~\ref{sec:eventReconstruction}.
In addition, they are required to pass category-specific $\pt$ thresholds.
The latter are motivated by the trigger selection.
Further requirements are placed on the charge-sum of leptons and $\tauh$
and, in most categories, also on the missing transverse momentum (measured with \metLD), 
and the multiplicity of AK4 or AK8 jets (targeting hadronic $\PW$ decays). 

The leading and subleading lepton in the \llss category must pass $\pt$ cuts of $25$ and $15\GeV$, respectively,
and have the same charge.
The events selected in this category are furthermore required to contain at least three AK4 jets or one AK8 jet. %% (TODO - check!)
Events in with both leptons are electrons must have $\metLD > 30\GeV$ to suppress
charge mis-identified $\Zee$ background.
After this selection, the main backgrounds in the \llss category are due to $\PW\PZ$ production
and to events in which one or both reconstructed leptons arise from the misidentification of a non-prompt lepton or hadron,
as shown in Table~\ref{tab:event_yields}.
The \WWWW decay mode accounts for roughly $80\%$ of
selected SM di-Higgs signal events, with \WWtt events accounting for the other $20\%$.

In the \lllnot category, 
the leading, subleading, and third lepton are required to
have $\pt$ values greater than $25$, $15$, and $10\GeV$, respectively, and the sum of their charges must be either plus or minus one.
At least one AK4 or AK8 jet must be present, and the $\metLD$ quantity must be greater than $30\GeV$,
or $45\GeV$ if there is a least one pair of same-flavor, opposite-charge ``fakeable''
leptons in the event.  Again backgrounds are dominated by $\PW\PZ$ production and events with misidentified
leptons, and the signal composition is similar to \llss.

The \llll category has identical lepton criteria, except that the third lepton
must have $\pt > 15\GeV$, and a fourth lepton with $\pt > 10\GeV$ is required, with
the total charge summing to zero.  Two thirds of signal events come from the \WWWW final
state, and one third from \WWtt, with $\PZ\PZ$ accounting for $85\%$ of the background.

The \lllt category matches the \lllnot requirements on the leptons, 
but in addition the presence of a $\tauh$ of charge opposite to the sum of the lepton charges is required.  
Almost $70\%$ of signal events come from $\HH \to \WWtt$,
and $\PZ\PZ$ production dominates the background.  

The \lltt categoriy requires
the leading and subleading lepton to pass $\pt$ thresholds $25$ and $15\GeV$, 
and the two $\tauh$ candidates with highest $\pt$ must bring the total lepton+$\tauh$ charge sum to zero.
Signal events are split $60$/$40\%$ between the \WWtt and \tttt final
states, and backgrounds come mostly from $\PZ\PZ$ production and from events with a misidentified lepton or $\tauh$.

To match trigger requirements, the leading lepton in \lttt events must have
$\abs{\eta} < 2.1$, and $\pt > 20\GeV$ if it is an electron and $\pt > 15\GeV$ if it is a muon.
The leading, subleading, and third
$\tauh$ must have $\pt > 40$, $30$, and $20\GeV$, respectively, and the sum of $\tauh$ and lepton charges must be zero.
Background events containing a $\Zee$ decay where one electron is mis-identified
as a $\tauh$ are vetoed by discarding events with an opposite-charge electron-$\tauh$
pair with mass between $71$ and $101\GeV$, if the $\tauh$ either fails a loose cut on the discriminant that separates $\tauh$ from electrons
or falls near the ECAL barrel-endcap ``crack'' in $1.46 < \abs{\eta} < 1.558$.  Around $80\%$ of selected signal events come
from the \tttt final state, with $20\%$ from \WWtt events, and most
background events coming from $\PZ\PZ$ production or containing a misidentified lepton or $\tauh$.

The \noltttt category has identical $\pt$ requirements on the leading three $\tauh$,
with the fourth \tauh also required to have $\pt > 20\GeV$.  Due to the extremely
low backgrounds, no charge sum criterion or $\Zee$ veto is applied.  Almost
all signal events come from the \tttt final state, with two thirds of the
background containing at least one fake $\tauh$, and the rest coming from $\PZ\PZ$ or single
Higgs production.

\begin{table}[!h]
\begin{center}
\begin{tabular}{|c|c|}

\hline
TO & DO \\
\hline

\end{tabular}
\end{center}
\caption{
  Event yields.
}
\label{tab:event_yields}
\end{table}

