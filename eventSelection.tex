\section{Event selection}
\label{sec:eventSelection}

Events are selected with the aim of maximizing the acceptance for $\HH$ decays to $\WWWW$, $\WWtt$, and
$\tttt$, while simultaneously rejecting the large backgrounds from multijet production,
the production of single $\PW$ and $\PZ$ bosons, $\PW$ and $\PZ$ boson pair production, and $\ttbar$ production.  To achieve this, we require multiple
reconstructed leptons $\lep$ (electrons or muons) or hadronically decaying taus
\tauh to be associated with the primary collision vertex.  The leptons and $\tauh$ may originate
from the decay of a $\PW$ boson or from a $\Pgt$ lepton decay.  Signal candidate events
are sub-divided into seven mutually exclusive categories based on lepton and $\tauh$
multiplicity: two same-sign leptons with less than two hadronic taus (\llss),
three leptons with no hadronic taus (\lllnot), at least four leptons (\llll),
three leptons with at least one additional hadronic tau (\lllt), two
leptons with at least two hadronic taus (\lltt), one lepton with at least
three hadronic taus (\lttt), or four or more hadronic taus with no leptons (\noltttt).
The event category is determined using lepton and $\tauh$ multiplicity based on the
``fakeable'' identification criteria described in Section~\ref{sec:eventReconstruction}.

To reduce background contamination from processes with top quarks, events with
at least one AK4 jet passing the ``medium'' b-tag identification, or at least two
jets passing the ``loose'' b-tag criteria, are discarded. 
Leptons originating from low-mass Drell-Yan production, decays of $\PJpsi$ and $\PUpsilon$ mesons,
cascade decays of bottom quarks, and photon conversions
are removed by vetoing events containing a pair of ``loose'' leptons with mass $m(\leplep) < 12\GeV$.
To eliminate overlap with events selected in the search for $\HH$ production in the $\Pbottom\APbottom\PZ\PZ$, $\PZ\PZ \to 4\Plepton$ decay mode~\cite{HIG-20-004}, no event in the
non-\tauh categories (\llss, \lllnot, and \llll) may contain two same-flavor,
opposite-charge ``loose'' lepton pairs with a combined mass less than $140\GeV$.
In addition, to reduce the \Zll background, these categories along with
\lltt exclude events where any same-flavor opposite-charge pair of ``loose''
leptons has a mass between $81$ and $101\GeV$.  %% Not applied in 3L1T? -AWB 11.02.2021

The leptons and \tauh selected in any of the signal regions of the seven event categories must pass the ``tight''
identification criteria described in Section~\ref{sec:eventReconstruction}.
In addition, they are required to pass category-specific $\pt$ thresholds,
which are motivated by the trigger selection.
Further requirements are placed on the charge-sum of leptons and $\tauh$,
and in most categories, on the missing transverse momentum (measured with $\metLD$),
and the multiplicity of AK4 or AK8 jets (targeting hadronic $\PW$ decays). 

The leading and subleading lepton in the \llss category must pass $\pt$ cuts of $25$ and $15\GeV$, respectively,
and have the same charge.
The events selected in this category are furthermore required to contain either at least two AK4 jets or one AK8 jet. %% TODO - check! -AWB 11.02.2021
Events in which both leptons are electrons must have $\metLD > 30\GeV$ and $m(\leplep) < 81$ or $m(\leplep) > 101$, in order to suppress
charge mis-identified $\Zee$ background.
After this selection, the main backgrounds in the \llss category are due to $\PW\PZ$ production
and to events in which one or both reconstructed leptons arise from the misidentification of a non-prompt lepton or hadron,
as shown in Table~\ref{tab:event_yields}.
The \WWWW decay mode accounts for roughly $80\%$ of
SM $\HH$ signal events selected in the \llss category, with \WWtt events accounting for the other $20\%$.

In the \lllnot category, 
the leading, subleading, and third lepton are required to
have $\pt$ values greater than $25$, $15$, and $10\GeV$, respectively, and the sum of their charges must be either plus or minus one.
At least one AK4 or AK8 jet must be present, and the $\metLD$ quantity must be greater than $30\GeV$,
or $45\GeV$ if there is a least one pair of same-flavor, opposite-charge ``fakeable''
leptons in the event.  Again backgrounds are dominated by $\PW\PZ$ production and events with misidentified
leptons, and the signal composition is similar to the \llss category.

The \llll category has identical lepton criteria, except that the third lepton
must have $\pt > 15\GeV$, and a fourth lepton with $\pt > 10\GeV$ is required, with
the total charge summing to zero.  Two thirds of signal events come from the \WWWW decay mode, 
and one third from \WWtt, while $\PZ\PZ$ production accounts for $85\%$ of the background.

The \lllt category matches the \lllnot requirements on the leptons, 
but in addition the presence of a $\tauh$ of charge opposite to the sum of the lepton charges is required.  
Background events in which the reconstructed $\tauh$ is due to the misidentification of an electron
are removed by vetoing events in which the $\tauh$ fails a loose cut on the discriminant that separates $\tauh$ from electrons
or falls near the ECAL barrel-endcap ``crack'' in $1.46 < \abs{\eta} < 1.558$.
Almost $70\%$ of signal events come from the $\WWtt$ decay mode,
while $\PZ\PZ$ production dominates the background.  

The \lltt category requires
the leading and subleading lepton to pass $\pt$ thresholds of $25$ and $15\GeV$, 
while the two $\tauh$ of highest $\pt$ must bring the total lepton plus $\tauh$ charge sum to zero.
Signal contributions are mostly due to the \WWtt ($60\%$) and \tttt ($40\%$) decay modes, 
while background contributions mostly arise from $\PZ\PZ$ production and from events with a misidentified lepton or $\tauh$.

In the \lttt category,
the leading lepton is required to satisfy the conditions 
$\abs{\eta} < 2.1$, and $\pt > 20 (15)\GeV$ if it is an electron (muon),
in order to pass the trigger.
The leading, subleading, and third
$\tauh$ must have $\pt > 40$, $30$, and $20\GeV$, respectively, and the sum of $\tauh$ and lepton charges is required to be zero.
Background events containing a $\Zee$ decay where one electron is misidentified
as a $\tauh$ are vetoed by discarding events containing an electron-$\tauh$ pair of opposite charge
and mass between $71$ and $101\GeV$, and in which the $\tauh$ either fails a loose cut on the discriminant that separates $\tauh$ from electrons,
or falls near the ECAL barrel-endcap ``crack'' in $1.46 < \abs{\eta} < 1.558$.  Around $80\%$ of $\HH$ signal events selected in the \lttt category
come from the \tttt and $20\%$ from the \WWtt decay mode, while the majority of
background events result from $\PZ\PZ$ production or contain a misidentified lepton or $\tauh$.

The \noltttt category has identical $\pt$ requirements on the leading three $\tauh$,
and the fourth \tauh is also required to have $\pt > 20\GeV$.  Due to the extremely
low backgrounds in this category, no charge sum criterion or $\Zee$ veto is applied.  Almost
all signal events come from the \tttt decay mode, while the background is dominated (at the level of $65\%$) by events 
containing at least one fake $\tauh$, and the remaining background contributions arise from $\PZ\PZ$ and single
$\PHiggs$ boson production.

\begin{table}[!h]
\begin{center}
\begin{scriptsize}
\begin{tabular}{lrlrlrl}
\hline
Process & \multicolumn{2}{c}{\llss} & \multicolumn{2}{c}{\lllnot} & \multicolumn{2}{c}{\llll} \\
\hline
$\HH \rightarrow \WWWW \times 30 \: \dagger$ & 39.0 & $\pm \: 0.9$ & 31.7 & $\pm \: 0.6$ & 2.15 & $\pm \: 0.09$ \\
$\HH \rightarrow \WWtt \times 30 \: \dagger$ & 9.2 & $\pm \: 0.3$ & 11.4 & $\pm \: 0.4$ & 0.90 & $\pm \: 0.06$ \\
$\HH \rightarrow \tttt \times 30 \: \dagger$ & 0.5 & $\pm \: 0.0$ & 0.9 & $\pm \: 0.0$ & 0.10 & $\pm \: 0.005$ \\
\hline
$\PZ\PZ$ & 26.0 & $\pm \: 0.8$ & 117.9 & $\pm \: 3.4$ & 54.85 & $\pm \: 2.32$ \\
$\PW\PZ$ & 545.1 & $\pm \: 12.4$ & 1334.5 & $\pm \: 14.6$ & 0.43 & $\pm \: 0.04$ \\
Conversion & 263.0 & $\pm \: 46.8$ & 156.0 & $\pm \: 29.6$ & 0.49 & $\pm \: 0.45$ \\
Fakes & 1433.6 & $\pm \: 53.4$ & 578.2 & $\pm \: 31.4$ & 2.40 & $\pm \: 1.03$ \\
Flips & 113.4 & $\pm \: 11.2$ &   ---  &  &   ---  & \\
Single Higgs & 77.4 & $\pm \: 2.3$ & 62.4 & $\pm \: 2.4$ & 2.48 & $\pm \: 0.34$ \\
Other backgrounds & 1117.9 & $\pm \: 56.1$ & 354.2 & $\pm \: 27.7$ & 4.48 & $\pm \: 0.35$ \\
Total expected background    & 3576.4 & $\pm \: 53.4$ & 2603.2 & $\pm \: 41.4$ & 65.13 & $\pm \: 2.02$ \\
\hline
Data &   ---  &  &   ---  &  &   ---  &  \\
\hline
\end{tabular}
\end{scriptsize}
\end{center}
\begin{center}
\begin{scriptsize}
\begin{tabular}{lrlrlrlrl}
\hline
Process & \multicolumn{2}{c}{\noltttt} & \multicolumn{2}{c}{\lttt} & \multicolumn{2}{c}{\lltt} & \multicolumn{2}{c}{\lllt} \\
\hline
$\HH \rightarrow \WWWW \times 30 \: \dagger$ & 0.27 & $\pm \: 0.003$ & 0.20 & $\pm \: 0.01$ & 0.16 & $\pm \: 0.01$ & 0.88 & $\pm \: 0.04$ \\
$\HH \rightarrow \WWtt \times 30 \: \dagger$ & 0.08 & $\pm \: 0.02$ & 0.59 & $\pm \: 0.11$ & 3.81 & $\pm \: 0.32$ & 3.95 & $\pm \: 0.17$ \\
$\HH \rightarrow \tttt \times 30 \: \dagger$ & 1.32 & $\pm \: 0.20$ & 2.49 & $\pm \: 0.27$ & 2.21 & $\pm \: 0.13$ & 0.85 & $\pm \: 0.03$ \\
\hline
$\PZ\PZ$ & 0.79 & $\pm \: 0.07$ & 1.89 & $\pm \: 0.12$ & 18.42 & $\pm \: 0.57$ & 24.14 & $\pm \: 0.41$ \\
$\PW\PZ$ &   ---  &  &   ---  &  & 0.01 & $\pm \: 0.003$ & 0.17 & $\pm \: 0.02$ \\
Fakes & 1.66 & $\pm \: 2.52$ & 2.53 & $\pm \: 2.32$ & 32.99 & $\pm \: 3.80$ & 27.36 & $\pm \: 3.29$ \\
Single Higgs & 0.38 & $\pm \: 0.14$ & 0.74 & $\pm \: 0.78$ & 2.94 & $\pm \: 1.18$ & 3.55 & $\pm \: 0.15$ \\
Other backgrounds & 0.03 & $\pm \: 0.05$ & 0.13 & $\pm \: 0.08$ & 2.63 & $\pm \: 0.47$ & 3.08 & $\pm \: 0.20$ \\
Total expected background    & 2.86 & $\pm \: 2.53$ & 5.29 & $\pm \: 2.48$ & 56.99 & $\pm \: 4.14$ & 58.30 & $\pm \: 3.38$ \\
\hline
Data &   ---  &  &   ---  &  &   ---  &  &   ---  &  \\
\hline
\end{tabular}
\end{scriptsize}
\end{center}
\begin{center}
\begin{scriptsize}
\begin{tabular}{lrlrl}
\hline
Process & \multicolumn{2}{c}{\threeLeptonCR} & \multicolumn{2}{c}{\fourLeptonCR}  \\
\hline
$\PZ\PZ$ & 805.6 & $\pm \: 9.1$ & 2030.9 & $\pm \: 28.5$ \\
$\PW\PZ$ & 12480.4 & $\pm \: 52.6$ & 0.5 & $\pm \: 0.2$ \\
Conversion & 140.4 & $\pm \: 24.5$ &   ---  & \\
Fakes & 864.6 & $\pm \: 24.7$ & 14.6 & $\pm \: 3.7$ \\
Other backgrounds & 612.0 & $\pm \: 33.3$ & 47.9 & $\pm \: 1.9$ \\
Total expected background    & 15001.7 & $\pm \: 72.6$ & 2114.4 & $\pm \: 14.8$ \\
\hline
Data & \multicolumn{2}{c}{14994} & \multicolumn{2}{c}{2118} \\
\hline
\end{tabular}
\\
\begin{tabular}{l}
  $\dagger$ The $\HH$ signal represents the sum of the \ggHH and \qqHH production processes \\
            and is normalized to $30 \times$ the event yield expected in the SM \\
\end{tabular}
\end{scriptsize}
\end{center}
\caption{
  Number of events selected in each of the seven search categories
  and in the two control regions for the dominant irreducible $\PW\PZ$ and $\PZ\PZ$ backgrounds,
  referred to as the ``\threeLeptonCR'' and ``\fourLeptonCR'', respectively, and described in Section~\ref{sec:backgroundEstimation}.
  The $\HH$ signal represents the sum of the $\ggHH$ and $\qqHH$ production processes and is normalized to $30$ the event yield expected in the SM.
  The expected event yields are computed for the values of nuisance parameters obtained from the ML fit described in Section~\ref{sec:results}.
  Quoted uncertainties represent the sum of statistical and systematic components. 
  A hyphen (—) indicates events yields that are smaller than the least significant digit.
}
\label{tab:event_yields}
\end{table}

