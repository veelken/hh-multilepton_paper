\section{Event selection}
\label{sec:eventSelection}

Events are selected which maximize acceptance of HH decays to \WWWW, \WWtt, and
\tttt while rejecting the overwhelming backgrounds from QCD, single W
boson, Z boson, and \ttbar production.  To achieve this, we require multiple
reconstructed leptons \lep (electrons or muons) or hadronically decaying taus
\tauh to be associated with the collision vertex in each event.  These may come
from the prompt decay of a W boson or from \Pgt decay.  Signal candidate events
are sub-divided into seven mutually exclusive categories based on lepton and \tauh
multiplicity: two same-sign leptons with less than two hadronic taus (\llss),
three leptons with no hadronic taus (\lllnot), at least four leptons (\llll),
three leptons with at least one additional hadronic tau (\lllt), two
leptons with at least two hadronic taus (\lltt), one lepton with at least
three hadronic taus (\lttt), or four or more hadronic taus (\noltttt).
The event category is determined using lepton and \tauh multiplicity based on the
``fakeable'' identification criteria described in Sec.~\ref{sec:eventReconstruction}.

To reduce background contamination from processes with top quarks, events with
at least one AK4 jet passing the ``medium'' b-tag identification, or at least two
jets passing the ``loose'' b-tag criteria, are discarded.  Since photon conversions
and light hadron decays can produce low-mass lepton pairs, all events containing a
``loose'' lepton pair with invariant mass $m(\leplep) < 12$~GeV are also removed.
To eliminate overlap with $\HH \to \PQb\PQb\PZ\PZ$ signal candidate events, no event in the
non-\tauh categories (\llss, \lllnot, and \llll) may contain two same-flavor,
opposite-charge ``loose'' lepton pairs with a combined invariant mass less than 140~GeV.
In addition, to reduce the \Zll background, these categories along with
\lltt exclude events where any same-flavor opposite-charge pair of ``loose''
leptons has an invariant mass between 81 and 101~GeV.


All leptons used to select signal candidate events must pass the ``tight'' threshold
for discrimination against non-prompt and fake leptons described in
Sec.~\ref{sec:eventReconstruction}, while \tauh objects must pass the ``medium'' ID,
except in the \lllt events where the ``tight'' threshold is more optimal. 
In most categories, additional requirements are placed on lepton and \tauh \pt
and charge, as well as missing energy (measured with \metLD), and the multiplicity
of AK4 or AK8 jets (targeting hadronic W decays).  Some categories also have dedicated
criteria to suppress \Zll and \Ztt backgrounds.

The two leading leptons in the \llss category must have the same charge and pass
\pt cuts of 25 and 15~GeV, respectively.  At least three AK4 jets or one AK8 jet %% (TODO - check!)
are also required, and di-electron events must have $\metLD > 30$~GeV to suppress
charge mis-identified \Zee background.  After selection, the main \llss
backgrounds are events with at least one non-prompt lepton and WZ~$\to$~3\lep, as shown
in Table~\ref{tab:event_yields}. The \WWWW decay mode accounts for roughly 80\% of
selected SM di-Higgs signal events, with \WWtt events accounting for the other 20\%.

In the \lllnot category, the three leading lepton charges must add up to $\pm1$, and
have \pt values greater than 25, 15, and 10~GeV, respectively.  At least one AK4
or AK8 jet must be present, and the \metLD quantity must be greater than 30~GeV,
or 45~GeV if there is a least one pair of same-flavor, opposite-charge ``fakeable''
leptons in the event.  Again backgrounds are dominated by WZ~$\to$~3\lep and non-prompt
leptons, and the signal composition is similar to \llss.
The \llll category has identical lepton criteria, except that the third lepton
must have $\pt > 15$~GeV, and a fourth lepton with $\pt > 10$~GeV is required, with
the total charge summing to 0.  Two thirds of signal events come from the \WWWW final
state, and one third from \WWtt, with ZZ accounting for 85\% of the background.

The \lllt category matches the \lllnot requirements on the leading three
leptons, but in addition the leading \tauh must have a charge opposite to the sum
of the lepton charges.  Almost 70\% of signal events come from HH~$\to \WWtt$,
and ZZ production dominates the background.  The \lltt categoriy requires
the two leading leptons to pass \pt thresholds 25 and 15~GeV, and the two \tauh
candidates with highest \pt must bring the total lepton+\tauh charge sum to 0.
Signal events are split 60/40\% between the \WWtt and \tttt final
states, and backgrounds come mostly from events with a non-prompt lepton or ZZ.

To match trigger requirements, the leading lepton in \lttt events must have
$|\eta| < 2.1$, with electron $\pt > 20$~GeV or muon $\pt > 15$~GeV.  The leading
three \tauh must have $\pt > 40$, 30, and 20~GeV, and have a charge sum of $\pm1$.
Background events containing a \Zee decay where one electron is mis-identified
as a \tauh are vetoed by discarding events with an opposite-charge electron-\tauh
pair with invariant mass between 71 and 101~GeV, if the \tauh fails the ``very-loose''
discriminanator criteria against electrons or falls near the ECAL barrel-endcap
``crack'' in $1.46 < |\eta| < 1.558$.  Around 80\% of selected signal events come
from the \tttt final state, with 20\% from \WWtt events, and most
background events containing a non-prompt or fake lepton, or coming from ZZ production.
The \noltttt category has identical \pt requirements on the leading three \tauh,
with the fourth \tauh also required to have $\pt > 20$~GeV.  Due to the extremely
low backgrounds, no charge sum criterion or \Zee veto is applied.  Almost
all signal events come from the \tttt final state, with two thirds of the
background containing at least one fake \tauh, and the rest coming from ZZ or single
Higgs production.

\begin{table}[!h]
\begin{center}
\begin{tabular}{|c|c|}

\hline
TO & DO \\
\hline

\end{tabular}
\end{center}
\caption{
  Event yields.
}
\label{tab:event_yields}
\end{table}

