\section{Systematic uncertainties}
\label{sec:systematicUncertainties}

When measuring the $\HH$ cross section for various signal models,
there are multiple sources of systematic uncertainty which affect
the predicted yields and BDT discriminator output shapes for different
sources of signal and background.
These uncertainties are either theoretical, affecting the predicted cross
section or decay kinematics of the collision process, or experimental,
accounting for possible mis-modeling of reconstructed objects.
Systematic uncertainties may either be treated as correlated or
uncorrelated across the three years of data taking, and among the
different signal and background processes.

The SM prediction for the inclusive $\ggHH$ production cross
section at $13\TeV$ center-of-mass collision energy is
$31.05^{+1.14}_{-1.196}\fb$ at NNLO accuracy, with missing higher
orders and undertainties in the proton parton density function (PDF)
and strong coupling $\alpha_{s}$ driving the uncertainty. %%~\cite{FIXME}
The Vector $\qqHH$ process has a cross section of $1.276\pm0.027~\fb$,
with PDF and $\alpha_{s}$ accounting for most of the uncertainty. %%~\cite{FIXME}
The predicted Higgs boson decay branching fractions to $\PW\PW$,
$\PGt\PGt$, and $\PZ\PZ$ have relative uncertainties of $1.54\%$,
$1.65\%$, and $1.54\%$, respectively. %%~\cite{FIXME}
%% https://github.com/HEP-KBFI/CombineHarvester/blob/master/ttH_htt/configs/list_syst_HH.py#L142-L161
Alternate $\HH$ predictions are generated with the renormalization
and factorization scales varied up and down by a factor of two, which
constitute an uncertainty on the BDT output shape and event yields
in each signal category, due to decay product acceptance in the detector.
All theoretical uncertainties on signal $\HH$ production are correlated
across all three years and among the selected event categories.
Their primary impact is on the measurement of the $\HH$ production
cross section as a ratio of the SM prediction.
Only the acceptance uncertainty affects the $\HH$ inclusive cross
section measurement, and none of these uncertainties impact the
limits on resonant $\HH$ production or EFT scenarios.
%% NOTES: Branching uncertainties in AN definitely incorrect, need to
%%        confirm correlation scheme, not clear if norm/fact is a "shape-only"
%%        uncertainty given that it is included in inclusive xsec uncertainty.
%%        Double-check SM mu vs. SM xsec vs. BSM usage of uncertainties.

Background processes which produce enough ``prompt'' electrons, muons,
and $\PGt$ leptons to match the selection criteria in one or more
event categories have theoretial uncertainties associated with their
production cross sections.
The relative uncertainties in the dominant $\PW\PZ$ and $\PZ\PZ$ background
cross sections are $2.1\%$ and $6.3\%$, respectively. %% ~\cite{FIXME}
Sub-dominant single Higgs production cross section uncertainties range
from $2$-$9\%$ for gluon fusion, VBF, and associated production with a
$\PW$ or $\PZ$ boson. %% ~\cite{FIXME}
Reducible backgrounds with one or two top quarks produced in association
with a $\PW$, $\PZ$, or $\PH$ boson have cross section uncertainties that
range from $8$ to $15\%$. %% ~\cite{FIXME}
The normalization for extremely rare backgrounds not mentioned above
(e.g.~triple boson or four top quark production) is given a $50\%$ uncertainty.
Background events where at least one reconstructed lepton or $\tauh$
originates from a photon conversion are assigned a $30\%$ yield uncertainty.
Theoretical uncertainties affecting background cross sections are treated
as uncorrelated among different physics processes, but correlated across
the different years and among all seven signal categories.
%% NOTES: Confirm correlation scheme outlined above.  Are "photon conversion"
%%        events all lumped together regardless of underlying physics process?

The prediction for events with at least one selected non-prompt,``fake'',
or charge mis-assigned lepton or $\tauh$ entering the signal region
(described in Sec.~\ref{sec:backgroundEstimation} is assigned a $30\%$
normalization uncertainty in most categories.
Since the $\lllt$ and $\lttt$ categories use modified $\tauh$ selection criteria
(Sec.~\ref{sec:eventReconstruction}), this uncertainty is raised to $50\%$.
An additional uncertainty on the BDT output shape in each category is
evaluated for events with a non-prompt or ``fake'' lepton or $\tauh$
by means of a ``closure test'' in MC simulation.
Events passing all signal selection criteria are compared to those
with at least one lepton or $\tauh$ failing the ``tight'' identification,
scaled according to the ``fakeable-to-tight'' method described in
Sec.~\ref{sec:backgroundEstimation}.
The ratio of these two shapes is fit with a linear function, which
is convoluted with the non-prompt and ``fake'' background prediction
from the data to serve as an uncertainty on the BDT output shape
for these events in the signal region.
The systematic uncertainties associated with the non-prompt and ``fake''
lepton prediction are treated as uncorrelated amongst the different
years, while the uncertainty on charge mis-assignment probability
is correlated.
%% NOTES: Are non-prompt yields correlated at all between categories?
%%        State level of closure in fake/flip yields and shape from MC closure, control regions?

Uncertainties in the modeling of the trigger and object reconstruction
affect all signal and background processes estimated using MC simulation.
Trigger efficiencies for events with at least two leptons are compared between
data and MC simulation in multi-lepton control regions enriched in the $\ttbar$,
$\PW\PZ$, and $\PZ\PZ$ background processes, as a function of lepton flavor,
$\pt$, and $\eta$.
This results in a small $\pt$ dependent uncertainty correlated between the
$\llss$ and $\lltt$ categories, and a $1\%$ normalization uncertainty
correlated among the $\lllnot$, $\lllt$, and $\llll$ categories.
For the $\noltttt$ and $\lttt$ categories, the data-to-simulation agreement in
trigger efficiency is computed using an independent set of events in data,
and parameterized as a function of the $\tauh$ $\pt$, $\eta$, $\phi$, and
reconstructed decay mode. %%~\cite{FIXME}
The trigger uncertainties for these two categories are treated as uncorrelated.
All systematic uncertainties related to trigger modeling are correlated
across different physics processes, but uncorrelated amongst the three years.
%% NOTES: Check which control regions are used for leptons, and pt/eta/other dependence.
%%        Verify that trigger efficiency scale factors are not applied to non-prompt.

The uncertainties in reconstruction and identification efficiency for electrons,
muons, and $\tauh$ have been measured in $\PZ$ boson enriched regions in data
for each level of identification criteria (``loose'', ``fakeable'', and ``tight''),
and are applied to each event as a function of the lepton $\pt$ and $\eta$, or of
the $\tauh$ $\pt$ and reconstructed decay mode.
The reconstructed $\tauh$ energy has an uncertainty of around $1\%$, depending on
the year and reconstructed $\tau$ decay mode.
These uncertainties affect the predicted rate and BDT output shape for signal
and background, and are correlated among the different physics processes, but
uncorrelated across different years.

The jet energy scale and resolution have uncertainties computed in common for
all CMS analyses, using di-jet control regions. %%~cite{FIXME}
The jet energy scale is evaluated using 11 separate components, some of which
are correlated across different years, and some uncorrelated.
The jet energy resolution uncertainty is uncorrelated among the three years.
Jet energy uncertainties are also propagated to the \metET calculation.
An additional uncertainty on the \metET vector comes from uncertainty in
the energy of unclustered PF hadrons, which is uncorrelated across different years.
The probability for true b jets to fail the multivariate b tagging
criteria, or for jets from gluons or light flavored hadrons to be mis-tagged
as b jets, is compared in data and MC simulation in event regions which
are enriched in light or heavy flavor jets. %%~cite{FIXME}
The resulting uncertainty in data-to-simulation agreement affects the yields
and BDT output shapes of multiple physics processes.
The statistical component of this uncertainty is treated as uncorrelated across
different years, while other experimental sources are considered correlated.

The total integrated luminosity for data collected in 2016, 2017, and 2018
have uncertainties of $2.2\%$, $2.0\%$, and $1.5\%$, respectively. %%~\cite{FIXME}
The measured cross section for inelastic $\Pp\Pp$ collisions is varied by
$5\%$, which affects the number of pileup vertices, and thus impacts the
jet reconstruction and lepton isolation. %%~\cite{FIXME}
Finally, a very small fraction of data in 2016 and 2017 was lost due to
trigger ``pre-firing'' induced by timing shifts in the ECAL detector.
The uncertainty in this fraction affects predicted event yields from
all signal and background processes, and is uncorrelated between 2016
and 2017 data. %%~cite{FIXME}

Of the systematic uncertainties listed above, those which have the largest
correlations to the final measured $\HH$ signal strength (for SM production)
include the $\tauh$ identification efficiency ($8.4\%$), background rates
from MC simulation ($8.0\%$), background rates from mis-identified leptons
or $\tauh$ ($8.0\%$), and the theoretical uncertainty on $\HH$ production
cross section and decay branching fractions ($6.2\%$).
All other sources of uncertainty have correlations of $3\%$ or less with
the SM $\HH$ signal strength.
